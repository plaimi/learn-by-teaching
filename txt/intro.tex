The "Bottom Of the Pyramid" (BOP) is a term central to the development of
value-chains in emerging markets in developing countries. It is understood as
the largest, although poorest segment of the population. The key to economic 
development lies in activating this segment of the population, enabling them 
not only to use modern (digital) services, but to author them as 
well\cite{prahalad2009fortune}.

The inspiration for our idea comes from applying this general framework to the
field of education software. The economics of educational services are 
unfavourable to those who need it the most.

We improve the status quo by offering a novel system for composing e-learning 
modules that is easy to use by non-technical users, but expressive enough to 
benefit power users as well.

This improvement is a stepping stone towards turning e-learning authoring into 
an activity achievable by even school children. It is also a stepping stone 
towards stimulating learning by teaching, a stimuli school children find 
themselves largely deprived of in most situated learning environments, i.e.\ 
schools.
