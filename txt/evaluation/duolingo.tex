\subsection{Evaluation of Duolingo}

Duolingo\footnote{\url{https://www.duolingo.com}} is a solution that provides 
spaced, paced learning. Users are presented with different challenges, usually 
centered around words (for example their pronunciation), and are able to 
advance through a hierarchy of lessons. Upon completion of lessons, modules, 
levels etc., users are granted experience points, which can be spent buying 
more attempts, in-game apparel, and so on. On completing levels, the overall 
retention of the lesson is noted using a strength bar. As time goes by, this 
retention decreases, and the user is prompted to re-take lessons, not 
dissimilar to the system provided by Anki.

Users are able to perform simple tasks like translating individual words or
sentences, pronouncing them correctly (using a microphone), conjugating nouns
and verbs, and so on. These tasks usually come in sets, and such a set in 
turns constitutes a module or level in the game. However, progression need not 
be linear, and more proficient users can opt to perform tasks qualifying for 
more advanced modules early on if so desired.

When performing a task, the user has a given set of hearts or HP (hit points, 
an analogy from games). When they fail a task, the count is decremented. Users 
can also opt for timed lessons.

\subsubsection{Expressiveness}

Duolingo is proprietary software that does not offer a way for users to author 
their own assets, and instead seeks to keep them content by providing it with 
the program.

Duolingo allows users to translate texts to (im)prove their language
proficiency, and uses community filtering to ascertain the quality of the
translations. This is how the service pays for itself --- as a mechanical 
turk for natural languages, currently partnering with CNN and
Buzzfeed\cite{duolingobuzz}. It also provides language certification
services\cite{duolingocert}.

Users are able to provide feedback on tasks, for example whether a translation
is reasonable or too hard at this proficiency level.

Duolingo tracks metrics, like what lessons people struggle the most with, 
adopting a wholesome data-driven approach. This combined with a flexible 
userbase allows them to iterate quickly on the assets provided in the 
lessons\cite{duolingodatadriven}.

But in terms of actually contributing with learning material, DuoLingo has 
 nothing to offer.

\subsubsection{Ease of use}

Duolingo appears to be easy to use. Quick feedback and positive reinforcement 
makes it easy to start using the system. There aren't many options available, 
except for audio settings (and other preferences), which makes the user 
interface easy to use.

\subsubsection{Takeaway points}

\begin{itemize}
\item Quick and simple feedback with additive gains facilitate user engagement
  greatly.
\item Community can be used to enhance or ensure the quality of lessons.
\item A data-driven approach might foster good community relations and provide more
  relevant, better assets as well.
\end{itemize}
