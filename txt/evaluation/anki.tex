\subsection{Evaluation of Anki}

Anki is a system for spaced learning. It does so by organising facts (in a wide
sense, like foreign words or parts of a map) into individual cards, which in
part constitute a deck of facts about a given topic. The user plays through the
deck by going through each card and stating how easy it was to recall the fact.
The software then replays cards depending on how good or bad the recall was
until some threshold of recollection has been met. Bad recall would result in 
the card appearing more often, and vice versa. This algorithm is inspired by the
SuperMemo system written in the 1980s. The intuition is that we reinforce
learning by recalling something just as we are about to forget about it.

Users are allowed to share, amend, or extend existing decks. While Anki
primarily exists for the purpose of individual learning, the composable and
free software format allows for others to take part in the learning 
experience. The software has enjoyed great popularity amongst several groups, 
most notably learners of foreign languages.

Anki is available both as a desktop client and through a web interface
(AnkiWeb). While the latter easily allows for online sharing of decks,
it doesn't allow for importing them. This is because it only synchronises with
the desktop client. This highlights the challenges faced by desktop and web
clients with different capabilities, and the value of isomorphic
implementations.

The desktop client of Anki will therefore be the main focus of this evaluation,
as it is the only full implementation of the system. Similar software in the
same area are Supermemo and Fluxcards.

\subsubsection{Expressiveness}

Anki allows for the creation of (decks of) flashcards using standard HTML
components with CSS styling. Thus, users are able to enrich cards with
customised formatting, sound, and video. The users are also able to record their
own voice (and repeat sounds) during playthrough. This offers an interactive
experience to the end-user. Because HTML and CSS are relatively flexible and
well-documented technologies, and their usage by many other prominent parties,
this allows cards to be rather extensible at almost no added cost.

\subsubsection{Ease of use}

Because the technologies used by Anki are freely available and readily
documented, they constitute an open standard. This open container format makes
Anki composable, and since the technologies are rather common (at least HTML),
users can make relatively simple changes without any major obstacles.

However, Anki is considered to have a poor user interface interaction story 
for common scenarios\cite{pcworldanki}.

An example of a problematic story is the rather common case of wanting to share
a deck. This usually has to be done through AnkiWeb, a free online service, 
which allows the user to download new decks made by other users. However, no such
suggestion is made to the user. Thus, the social dimension of Anki is 
downplayed.

Similarly, when using the software for the first time, the user is greeted by 
a shallow window with a single deck. No explanation of Anki or its algorithm 
is given. The user can click a simple built-in test deck to play through it, 
but this is not suggested by the software, and the deck does not showcase 
Anki's capabilities either (for example the use of rich media or LaTeX 
equations).

Other examples include the main window, which has both a menu bar and some
buttons. One of the buttons suggests that the user might be able to import new
decks, but turns out to be about constructing new decks.

The most complex interface in Anki is the library view. It allows for somewhat
non-trivial querying of all the decks in use, with numerous built-in categories
and tags for distinguishing them (be it difficulty, topic, or the decks
scheduled for today). Also, this rich view is not offered in AnkiWeb, so there's
an inherent divide in the capabilities of searching already imported decks and
looking for new ones, limiting the ability to explore.

\subsubsection{Misc}

\begin{itemize}
\item AnkiWeb allows for synchronization and sharing of card decks (as files) using
  a free online service. It is also able to play through the decks, but not
  import them from stand-alone files.
\item Freely licensed under the terms of the GNU AGPL v3+. Sub-components are
  licensed under a variety of licenses, mostly GPL and flavours of BSD/MIT 
  licences.
\item All art assets are freely licensed. The logo in particular is also GNU AGPL
  v3+, but with a fair-use option for blogs, news outlets etc.
\item Universal design? % TODO
\end{itemize}

\subsubsection{Takeaway points}

\begin{itemize}
\item Anki offers spaced repetition learning, which might be an interesting 
  way to improve recall and strengthen comprehension.
\item The simple metaphor (a deck of flashcards) has been proven intuitive 
  (SOURCES
  XXX) and easy to implement, as well as explain to users. % TODO
\item Anki's use of HTML elements provides a universal container format and
  facilitates sharing. However, the web interface is very limited, and while it
  does allow for sharing, it does not allow for importing decks, thus not
  providing a full (isomorphic) user experience across platforms.
\item The desktop client has a problematic user interface story, and does not
  provide enough directions or examples for effective use. Modal dialogues and
  context switching might provide a more elegant solution for capturing common
  usage patterns (compared to displaying the full set of choices and categories
  available to the user, e.g.\ in the library view).
\end{itemize}
