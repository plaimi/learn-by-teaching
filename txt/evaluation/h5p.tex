\subsection{H5P}

H5P is a standard allowing for reusable, portable e-learning modules. It does 
so by providing plug-ins for Drupal\footnote{\url{https://www.drupal.org/}}, 
WordPress\footnote{\url{https://wordpress.org/}}, and 
Joomla\footnote{\url{https://www.joomla.org/}}, along with a file format 
specification for distribution, as well as standard \emph{content types} with 
built-in editors.

The value proposition of H5P is that it integrates nicely with existing
infrastructure like databases and view layers, enabling developers to focus on
developing their modules.\cite{h5pwhy} I.e., it reduces the ceremony needed
to get going. Moreover, its portability provides a much-wanted ease of
deployment for site owners, as well as a bigger market for developers.

Before deciding whether to use H5P we needed to evaluate five core issues.

\subsubsection{Are the abstractions offered by H5P powerful enough?}

At its very core, H5P offers a contract for libraries, which might function as
full-blown applications, stand-alone libraries, or application programming 
interfaces for any combination of the above. 

This is done in part by making the libraries document their semantics
(`semantics.json'), i.e.\ the properties (data structure) that are used to set
up the application. Each property is described by its name, as well as its type.
Different things have different semantic requirements: E.g., the
`list' (ordered group) semantic type requires the name of a single entity in it
(`entity'). This information is used by the generic editor to allow users to
easily customise the library.

H5P also provides a container format that manages ceremony, like loading
the appropriate CSS and JavaScript files, as well as providing useful metadata 
like the version and name of the library. This format includes a JavaScript 
class which libraries extend to hook into H5P. Applications are required to 
implement an `attach' method, which accepts the identity of the container in 
which to attach the module.

Of particular interest for us is the central event dispatcher 
(`H5P.EventDispatcher'), which is used to dispatch actions within 
H5P\cite{h5pdispatch}. Thus, like Elm's ports\cite{elmports} or Facebook 
React's Flux architecture/dispatcher\cite{fluxdispatch}, a synchronous 
interface for internal communication is provided, which might prove useful for 
multi-component systems.

Beyond this, H5P leaves the library developers plenty of room to expand. The 
H5P standard is pretty much ``open world'' and doesn't necessarily limit the 
structure of functionality of the JavaScript libraries it wraps, enabling 
developers to make arbitrarily complex software.

However, it should be noted that H5P still depends on professional
developers, and doesn't provide primitives that necessary make any sense for
ordinary users. Therefore, the standard per se doesn't preclude a more
user-oriented canvas system.

In conclusion, H5P doesn't itself provide the needed abstractions necessary, but
doesn't hinder them from being developed. We can use H5P for now.

\subsubsection{Are the abstractions high-level enough for our purposes?}

Given our emphasis on an intuitive user-interface for liberating the BOP, H5P 
quite frankly doesn't cut it for the time being. But even if H5P is not nice 
enough to use (at the moment) for non-technical users turned authors, it 
\emph{does} play nice with other software --- or, at the very least, it stays 
out of the way. This is \emph{very} desirable to us.

Once we have H5P modules ``up and running'', so to speak, there's nothing 
stopping us from moving ahead with other kinds of e-learning modules. There is 
furthermore nothing precluding H5P being improved in the future; we could even 
conceivably do that ourselves.

The honest truth is that H5P is not good enough for our intended target 
demographic. But it \emph{is} free software that integrates nicely and may be 
improved in the future. We'll take it.

\subsubsection{Can we generate valid H5P modules?}

Yes. Another short answer, but a more positive one at that.

As already noted, H5P doesn't limit any system wrapped in its container
format. Therefore, any functioning JavaScript system could in principle be
embedded within a H5P module, provided it only operates on a single node (due 
to the complexity and performance implications of coordinating DOM operations 
on overlapping elements).

\subsubsection{Can we import H5P modules?}

Again, yes! We can! We would need to make something that would be able to 
parse the semantic definitions of H5P modules, and we'd have to use this in 
some meaningful way. Effectively we should enhance the authoring tool as well 
to make it more appropriate for our target demographic.

\subsubsection{Can we offer a new way of composing H5P?}

H5P has no composition guarantees, and resorts to Design by Contract (DbC) for
composite systems.\cite{h5pdbc} This does nothing to ensure a uniform
communication standard, which ensures flexibility, but does not provide a
uniform interface. The specification itself admits that this is due to the lack
of interfaces in JavaScript. Thus, the contracts are more by politeness and not
by obligation, potentially creating weak implementations.

Therefore, there is obviously definite value to be provided by a system
providing more stringent, verifiable contracts between components, for example
using a more powerful type system than the one found in JavaScript. This might
in turn result in more reusable and robust applications. Also, because H5P 
effectively provides no composition contracts, this system might function as a 
sandbox and provide useful feedback should future versions of H5P seek to 
provide more formal requirements for composite modules.

One related problem is that H5P provides no standard for sub-module 
inheritance/reuse of content, as the system only has a notion of full H5P 
modules. We must figure out a way for several modules to refer to the same 
data.
