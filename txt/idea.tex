We suggest developing an engaging "canvas"  for authoring e-learning modules, 
taking cues from RPG\begin{CJK}{UTF8}{min}ツクール\end{CJK}, Game Maker, and 
other similar software that succeed in leveraging an intuitive interface with 
a capacious featureset. By "canvas" (or "scene graph"), we mean an 
easy-to-learn system of widgets and composite stores (with rich content), and 
flows connecting them together to a cohesive system. 

An example of such a system might be a quiz leveraging media like GNU 
Mediagoblin, Youtube, NRK, and Twitter to showcase the issue at hand, before 
testing the user's comprehension at the end. Thus, the 
student-come-software-developer is able to prove their comprehension of 
multimedia (as mandated by the curriculum), and other students allowed to take 
part in the learning experience. Incentives for doing so might be given by the 
teacher, or by arranging competitions that invite the students to share their 
best ideas.

Through use of our canvas, we eliminate the BOP by liberating and empowering 
it to make its own user experience. As an added bonus, our canvas may spark 
some latent creative souls, or inspire technological awareness and interest. 
In our increasingly computerised society, this is in itself a noble 
cause.Thus, this canvas might prove to be the tool needed to bridge the gap 
between just being a user and a future career in software, as there is no 
readily-available path for acquiring the advanced knowledge needed to develop 
modern systems given our current education system.

We aim to make our canvas the easiest to use way of authoring e-learning 
modules, and at the same time making it powerful enough to entice power users 
and established e-learning module authors. Consequently, the target 
demographic of our canvas is not limited to the BOP, but extends to include 
current e-learning module authors.

Initially we aim to support authoring H5P modules, focussing our development 
at H5P integration.
